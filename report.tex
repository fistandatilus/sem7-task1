
\documentclass[12pt,a4paper]{article}
\usepackage[utf8]{inputenc}
\usepackage[OT1]{fontenc}
\usepackage[russian]{babel}
\usepackage{amsmath}
\usepackage{amsfonts}
\usepackage{amssymb}
\usepackage{hyperref}
\hypersetup{linktoc=all}
\usepackage{indentfirst}
\usepackage[left=2cm,right=2cm,top=2cm,bottom=2cm]{geometry}
\usepackage{graphicx}
\usepackage{float}
\usepackage{pdflscape}
\usepackage{}
\author{Борисенков Никита Николаевич}
\title{Отчёт}
\date{}

\newcommand{\xo}{\mathring{x}}
\newcommand{\xx}{x\overline{x}}
\newcommand{\pd}[2]{\dfrac{\partial #1}{\partial #2}}
\DeclareMathOperator*{\mmax}{max}

\begin{document}
\maketitle
\tableofcontents
\newpage
\section{Постановка задачи}

Решается задача моделирования теплопроводного вязкого баротропного газа.

$$
\begin{array}{rcl}
    \pd{\rho}{t} + \pd{\rho u}{x} & = & 0\\
    \pd{\rho u}{t} + \pd{\rho u^2}{x} + \pd{p}{x} & = & \mu\pd{^2 u}{x^2} + \rho f
\end{array}
$$

\section{Алгоритм}

Используется последовательная схема с центральными разностями ($u, \rho$)

Уравнения, задающие схему выглядят следующим образом:
$$
\begin{array}{lc}
    V_t + \frac13 \left( V\hat{V}_{\xo} + ( V\hat{V} )_{\xo} \right) + \dfrac{p\left(H\right)_{\xo}}{H} = \tilde{\mu}\hat{V}_{\xx} - \left( \tilde{\mu} - \dfrac{\mu}{H} \right)V_{\xx} + f, & x \in \omega_h, \\

    H_t + 0,\!5 \left(  \hat{V}\hat{H}_{\xo} + (\hat{V}\hat{H})_{\xo} + H\hat{V}_{\xo}\right) = f_0, & x \in \omega_h, \\

    H_{t,0} + 0,\!5\left((\hat{V}\hat{H})_{x,0} + H_0\hat{V}_{x,0}\right) - & \\
    - 0,\!5h\left(\left(HV\right)_{\xx,1} - 0,\!5\left(HV\right)_{\xx,2} + H_0\left(V_{\xx,1} - 0,\!5V_{\xx,2}\right)\right) = f_0, & \\

    H_{t,M} + 0,\!5\left((\hat{V}\hat{H})_{\overline{x},M} + H_0\hat{V}_{\overline{x},M}\right) + & \\
    + 0,\!5h\left(\left(HV\right)_{\xx,M-1} - 0,\!5\left(HV\right)_{\xx,M-2} + H_0\left(V_{\xx,M-1} - 0,\!5V_{\xx,M-2}\right)\right) = f_0, & 
\end{array}
$$
где $\tilde{\mu} = \displaystyle\max_{m} \dfrac{\mu}{H}$.

По этим уравнениям сначала строится СЛУ на значения $V$ на следующеми слое, которое решается методом прогонки, а с использованием полученных значений $\hat{V}$ строится и решается СЛУ на $\hat{H}$.

После преобразований, СЛУ схемы имеют следующий вид:
\begin{gather*}
    V_{m-1}^{n+1}\left(-\dfrac{\tau\tilde{\mu}}{h^2}-\dfrac{\tau(V_{m}^n + V_{m-1}^n)}{6h}\right) + 
    V_{m}^{n+1}\left(1+\dfrac{2\tau\tilde{\mu}}{h^2}\right) +
    V_{m+1}^{n+1}\left(-\dfrac{\tau\tilde{\mu}}{h^2}+\dfrac{\tau(V_{m}^{n} + V_{m+1}^{n})}{6h} \right)= \\ =
    V_{m}^n-\dfrac{\tau(p(H_{m+1}^{n}) - p(H_{m-1}^n))}{2hH_m^n} -
    \left( \tilde{\mu} - \dfrac{\mu}{H_m^n} \right) \dfrac{V_{m-1}^n - 2V_{m}^n + V_{m+1}^n}{h^2} + \tau f_m^{n}
    , m = 1,\dots, M-1,\\
    V_0^{n+1} = 0, V_M^{n+1} = 0, \\
    H_{m-1}^{n+1}\dfrac{\tau}{4h}(-V_{m}^{n+1} - V_{m-1}^{n+1}) +
    H_{m}^{n+1} +
    H_{m+1}^{n+1}\dfrac{\tau}{4h}(V_{m}^{n+1} + V_{m+1}^{n+1}) = \\
    = H_{m}^{n} - \tau H_{m}^{n}\left( \dfrac{V_{m+1}^{n+1} - V_{m-1}^{m+1}}{2h} \right) + \tau f_{0m}^{\phantom{0}n+1}
    , m = 1,\dots, M-1,\\
    H_0^{n+1}\left( 1 - \dfrac{\tau V_0^{n+1}}{2h} \right) + H_1^{n+1}\dfrac{\tau V_1^{n+1}}{2h} = \\
    = H_0^{n} - \dfrac{\tau}{2h}(H_0^nV_1^{n+1} - H_0^nV_0^{n+1} - 2H_0^nV_0^n + 2.5H_1^nV_1^n - 2H_2^nV_2^n + 0.5H_3^nV_3^n + \\
    + 2.5H_0^nV_1^n - 2H_0^nV_2^n + 0.5H_0^nV_3^n) + \tau f_{0,0}^{\phantom{0}n+1}, \\
    H_{M-1}^{n+1}\left(-\dfrac{\tau V_{M-1}^{n+1}}{2h}\right) + H_M^{n+1}\left(1 + \dfrac{\tau V_M^{n+1}}{2h}\right) = \\
    = H_M^{n} - \dfrac{\tau}{2h}(H_M^nV_M^{n+1} - H_M^nV_{M-1}^{n+1} + 2H_M^nV_M^n - 2.5H_{M-1}^nV_{M-1}^n + 2H_{M-2}^nV_{M-2}^n - 0.5H_{M-3}^nV_{M-3}^n - \\
    - 2.5H_M^nV_{M-1}^n + 2H_M^nV_{M-2}^n - 0.5H_M^nV_{M-3}^n) + \tau f_{0,M}^{\phantom{0,}n+1}.
\end{gather*}

\include{task1}

\section{Негладкие начальные данные}

Решаются 2 задачи:
\begin{equation} \label{p1_1}
    \begin{array}{ll}
        \rho_0 = 1, & x < 4.5 \text{или} x > 5.5, \\
        \rho_0 = 2, & x \in [4.5; 5.5], \\
        u_0 \equiv 0, & x \in [0;10],\\
        u(t,0) = u(t, 10) = 0, & t \in [0; T].
    \end{array}
\end{equation}
\begin{equation} \label{p1_2}
    \begin{array}{ll}
        u_0 = 0, & x < 4.5 \text{или} x > 5.5, \\
        u_0 = 1, & x \in [4.5; 5.5], \\
        \rho_0 \equiv 1, & x \in [0;10],\\
        u(t,0) = u(t, 10) = 0, & t \in [0; T].
    \end{array}
\end{equation}

В обоих задачах функция $f$ равна 0.
\subsection{Первая задача}
Далее приведены таблицы со значениями нормы и изменения "массы". Сравнение по методу вложенных сеток проводилось на уровне $n_{st}/10$.
\subsubsection{$\mu = 0.1, p(\rho) = \rho $}
\begin{tabular}{*{6}{|l}|}
    \hline
    \multicolumn{6}{|c|}{$h = 0.01, \tau = 0.1$} \\
    \hline
$\|\cdot \|$& $1.311107e-01$ & $4.736235e-02$ & $2.147813e-02$ & $9.962049e-03$ &$181.740000$\\
\hline
$\triangle_{mass}$& $-1.782240e-03$ & $-1.905372e-03$ & $-1.892737e-03$ & $-1.898413e-03$ &\\
\hline

\end{tabular}

$\|v-v^{4}\|_{C_h} = 1.807288e-02$

\begin{tabular}{*{6}{|l}|}
    \hline
    \multicolumn{6}{|c|}{$h = 0.001, \tau = 0.1$} \\
    \hline
    &$n_{st}/4 $&$ n_{st}/2$&$3n_{st}/4$&$n_{st}$&$T_{st}$ \\
    \hline
$\|\cdot \|$& $9.738945e-02$ & $5.540849e-02$ & $2.627760e-02$ & $9.994081e-03$ &$166.730000$\\
\hline
$\triangle_{mass}$& $-1.905990e-03$ & $-1.972209e-03$ & $-1.986532e-03$ & $-1.990592e-03$ &\\
\hline
\end{tabular}

$\|v-v^{4}\|_{C_h} = 4.879192e-03$

\begin{tabular}{*{6}{|l}|}
    \hline
    \multicolumn{6}{|c|}{$h = 0.01, \tau = 0.0001$} \\
    \hline
    &$n_{st}/4 $&$ n_{st}/2$&$3n_{st}/4$&$n_{st}$&$T_{st}$ \\
    \hline
    $\|\cdot \|$& $1.330767e-01$ & $4.767380e-02$ & $2.148564e-02$ & $9.997596e-03$ &$181.745000$\\
\hline
$\triangle_{mass}$& $-6.142264e-05$ & $-1.268190e-04$ & $-1.053240e-04$ & $-1.089084e-04$ &\\
\hline
\end{tabular}

$\|v-v^{3}\|_{C_h} = 1.575023e-01$

\begin{tabular}{*{6}{|l}|}
    \hline
    \multicolumn{6}{|c|}{$h = 0.0001, \tau = 0.00001$} \\
    \hline
    &$n_{st}/4 $&$ n_{st}/2$&$3n_{st}/4$&$n_{st}$&$T_{st}$ \\
    \hline
$\|\cdot \|$& $1.413131e-01$ & $5.613559e-02$ & $2.373457e-02$ & $9.999895e-03$ &$171.703100$\\
\hline
$\triangle_{mass}$& $-2.470201e-06$ & $-9.384597e-06$ & $-1.226752e-05$ & $-1.096265e-05$ &\\
\hline
\end{tabular}

$\|v-v^{2}\|_{C_h} = 1.614460e-02$

\subsubsection{$\mu = 0.1, p(\rho) = 10\rho $}

\begin{tabular}{*{6}{|l}|}
    \hline
    \multicolumn{6}{|c|}{$h = 0.01, \tau = 0.01$} \\
    \hline
    &$n_{st}/4 $&$ n_{st}/2$&$3n_{st}/4$&$n_{st}$&$T_{st}$ \\
    \hline
$\|\cdot \|$& $2.060665e-01$ & $6.177272e-02$ & $3.860089e-02$ & $9.895097e-03$ &$135.570000$\\
\hline
$\triangle_{mass}$& $-1.665168e-02$ & $-1.691972e-02$ & $-1.698062e-02$ & $-1.700048e-02$ &\\
\hline    
\end{tabular}

$\|v-v^{4}\|_{C_h} = 1.254366e-01$

\begin{tabular}{*{6}{|l}|}
    \hline
    \multicolumn{6}{|c|}{$h = 0.01, \tau = 0.001$} \\
    \hline
    &$n_{st}/4 $&$ n_{st}/2$&$3n_{st}/4$&$n_{st}$&$T_{st}$ \\
    \hline
$\|\cdot \|$& $1.261085e-01$ & $8.367589e-02$ & $4.037255e-02$ & $9.985122e-03$ &$137.087000$\\
\hline
$\triangle_{mass}$& $-1.879069e-03$ & $-1.880629e-03$ & $-1.874761e-03$ & $-1.874880e-03$ &\\
\hline
\end{tabular}

$\|v-v^{4}\|_{C_h} = 1.265462e-02$

\begin{tabular}{*{6}{|l}|}
    \hline
    \multicolumn{6}{|c|}{$h = 0.001, \tau = 0.01$} \\
    \hline
    &$n_{st}/4 $&$ n_{st}/2$&$3n_{st}/4$&$n_{st}$&$T_{st}$ \\
    \hline
    $\|\cdot \|$& $2.155009e-01$ & $8.756834e-02$ & $3.580089e-02$ & $9.785629e-03$ &$127.660000$\\
\hline
$\triangle_{mass}$& $-1.667108e-02$ & $-1.700650e-02$ & $-1.706233e-02$ & $-1.707678e-02$ &\\
\hline
\end{tabular}

$\|v-v^{4}\|_{C_h} = 1.431710e-01$

\begin{tabular}{*{6}{|l}|}
    \hline
    \multicolumn{6}{|c|}{$h = 0.001, \tau = 0.001$} \\
    \hline
    &$n_{st}/4 $&$ n_{st}/2$&$3n_{st}/4$&$n_{st}$&$T_{st}$ \\
    \hline
$\|\cdot \|$& $1.291632e-01$ & $8.777331e-02$ & $4.187694e-02$ & $9.987284e-03$ &$130.760000$\\
\hline
$\triangle_{mass}$& $-1.917629e-03$ & $-1.959441e-03$ & $-1.965297e-03$ & $-1.965237e-03$ &\\
\hline
\end{tabular}

$\|v-v^{3}\|_{C_h} = 1.347214e-02$

\subsubsection{$\mu = 0.1, p(\rho) = 100\rho $}

\begin{tabular}{*{6}{|l}|}
    \hline
    \multicolumn{6}{|c|}{$h = 0.1, \tau = 0.001$} \\
    \hline
    &$n_{st}/4 $&$ n_{st}/2$&$3n_{st}/4$&$n_{st}$&$T_{st}$ \\
    \hline
$\|\cdot \|$& $1.639018e-01$ & $6.124743e-02$ & $4.390264e-02$ & $9.985283e-03$ &$136.889000$\\
\hline
$\triangle_{mass}$& $-1.537634e-02$ & $-1.561764e-02$ & $-1.567686e-02$ & $-1.568670e-02$ &\\
\hline    
\end{tabular}

$\|v-v^{4}\|_{C_h} = 2.651876e-01$

\begin{tabular}{*{6}{|l}|}
    \hline
    \multicolumn{6}{|c|}{$h = 0.01, \tau = 0.001$} \\
    \hline
    &$n_{st}/4 $&$ n_{st}/2$&$3n_{st}/4$&$n_{st}$&$T_{st}$ \\
    \hline
    $\|\cdot \|$& $2.916086e-01$ & $1.092784e-01$ & $4.084209e-02$ & $9.993343e-03$ &$104.333000$\\
\hline
$\triangle_{mass}$& $-1.668026e-02$ & $-1.674483e-02$ & $-1.675986e-02$ & $-1.676538e-02$ &\\
\hline
\end{tabular}

$\|v-v^{4}\|_{C_h} = 3.150457e-01$


\begin{tabular}{*{6}{|l}|}
    \hline
    \multicolumn{6}{|c|}{$h = 0.001, \tau = 0.001$} \\
    \hline
    &$n_{st}/4 $&$ n_{st}/2$&$3n_{st}/4$&$n_{st}$&$T_{st}$ \\
    \hline
$\|\cdot \|$& $2.755433e-01$ & $1.040215e-01$ & $3.737523e-02$ & $9.985259e-03$ &$110.334000$\\
\hline
$\triangle_{mass}$& $-1.684067e-02$ & $-1.690602e-02$ & $-1.691713e-02$ & $-1.691904e-02$ &\\
\hline

\end{tabular}

$\|v-v^{2}\|_{C_h} = 2.303484e-01$

\subsubsection{$\mu = 0.1, p(\rho) = \rho^{1,4} $}

\begin{tabular}{*{6}{|l}|}
    \hline
    \multicolumn{6}{|c|}{$h = 0.1, \tau = 0.1$} \\
    \hline
    &$n_{st}/4 $&$ n_{st}/2$&$3n_{st}/4$&$n_{st}$&$T_{st}$ \\
    \hline
    $\|\cdot \|$& $9.868017e-02$ & $3.264695e-02$ & $1.522756e-02$ & $9.898022e-03$ &$215.700000$\\
\hline
$\triangle_{mass}$& $-2.389534e-02$ & $-2.495181e-02$ & $-2.480563e-02$ & $-2.491468e-02$ &\\
\hline
\end{tabular}

$\|v-v^{4}\|_{C_h} = 6.921404e-02$


\begin{tabular}{*{6}{|l}|}
    \hline
    \multicolumn{6}{|c|}{$h = 0.01, \tau = 0.01$} \\
    \hline
    &$n_{st}/4 $&$ n_{st}/2$&$3n_{st}/4$&$n_{st}$&$T_{st}$ \\
    \hline
$\|\cdot \|$& $1.130977e-01$ & $4.510049e-02$ & $2.202760e-02$ & $9.983975e-03$ &$177.360000$\\
\hline
$\triangle_{mass}$& $-2.864375e-03$ & $-3.014560e-03$ & $-3.028172e-03$ & $-3.016940e-03$ &\\
\hline
\end{tabular}

$\|v-v^{4}\|_{C_h} = 1.346759e-02$


\begin{tabular}{*{6}{|l}|}
    \hline
    \multicolumn{6}{|c|}{$h = 0.001, \tau = 0.01$} \\
    \hline
    &$n_{st}/4 $&$ n_{st}/2$&$3n_{st}/4$&$n_{st}$&$T_{st}$ \\
    \hline
    $\|\cdot \|$& $1.257300e-01$ & $5.087393e-02$ & $2.678942e-02$ & $9.980486e-03$ &$160.840000$\\
\hline
$\triangle_{mass}$& $-3.020199e-03$ & $-3.102258e-03$ & $-3.115073e-03$ & $-3.118401e-03$ &\\
\hline
\end{tabular}

$\|v-v^{4}\|_{C_h} = 1.105816e-02$

\begin{tabular}{*{6}{|l}|}
    \hline
    \multicolumn{6}{|c|}{$h = 0.01, \tau = 0.001$} \\
    \hline
    &$n_{st}/4 $&$ n_{st}/2$&$3n_{st}/4$&$n_{st}$&$T_{st}$ \\
    \hline
\end{tabular}

$\|v-v^{4}\|_{C_h} = 1.338673e-02$

\subsubsection{$\mu = 0.01, p(\rho) = 1\rho $}

\begin{tabular}{*{6}{|l}|}
    \hline
    \multicolumn{6}{|c|}{$h = 0.1, \tau = 0.01$} \\
    \hline
    &$n_{st}/4 $&$ n_{st}/2$&$3n_{st}/4$&$n_{st}$&$T_{st}$ \\
    \hline
$\|\cdot \|$& $4.475577e-02$ & $2.256459e-02$ & $1.391919e-02$ & $9.998176e-03$ &$1131.290000$\\
\hline
$\triangle_{mass}$& $-1.538956e-02$ & $-1.564784e-02$ & $-1.570870e-02$ & $-1.566975e-02$ &\\
\hline    
\end{tabular}

$\|v-v^{4}\|_{C_h} = 6.148983e-02$


\begin{tabular}{*{6}{|l}|}
    \hline
    \multicolumn{6}{|c|}{$h = 0.01, \tau = 0.01$} \\
    \hline
    &$n_{st}/4 $&$ n_{st}/2$&$3n_{st}/4$&$n_{st}$&$T_{st}$ \\
    \hline
$\|\cdot \|$& $7.450361e-02$ & $4.082204e-02$ & $1.729210e-02$ & $9.990536e-03$ &$630.840000$\\
\hline
$\triangle_{mass}$& $-1.651819e-02$ & $-1.670958e-02$ & $-1.673876e-02$ & $-1.676143e-02$ &\\
\hline
\end{tabular}

$\|v-v^{4}\|_{C_h} = 1.073211e-01$


\begin{tabular}{*{6}{|l}|}
    \hline
    \multicolumn{6}{|c|}{$ = 0.001, \tau = 0.01$} \\
    \hline
    &$n_{st}/4 $&$ n_{st}/2$&$3n_{st}/4$&$n_{st}$&$T_{st}$ \\
    \hline
    $\|\cdot \|$& $6.798890e-02$ & $3.183352e-02$ & $1.620904e-02$ & $9.972762e-03$ &$730.830000$\\
\hline
$\triangle_{mass}$& $-1.671689e-02$ & $-1.687954e-02$ & $-1.690742e-02$ & $-1.691478e-02$ &\\
\hline
\end{tabular}

$\|v-v^{3}\|_{C_h} = 8.499027e-02$


\begin{tabular}{*{6}{|l}|}
    \hline
    \multicolumn{6}{|c|}{$h = 0.01, \tau = 0.001$} \\
    \hline
    &$n_{st}/4 $&$ n_{st}/2$&$3n_{st}/4$&$n_{st}$&$T_{st}$ \\
    \hline
$\|\cdot \|$& $7.338187e-02$ & $3.978785e-02$ & $1.733478e-02$ & $9.999736e-03$ &$630.687000$\\
\hline
$\triangle_{mass}$& $-1.881417e-03$ & $-1.878079e-03$ & $-1.873083e-03$ & $-1.885347e-03$ &\\
\hline
\end{tabular}

$\|v-v^{3}\|_{C_h} = 1.278493e-02$

\subsubsection{$\mu = 0.01, p(\rho) = 10\rho $}

\begin{tabular}{*{6}{|l}|}
    \hline
    \multicolumn{6}{|c|}{$h = 0.01, \tau = 0.001$} \\
    \hline
    &$n_{st}/4 $&$ n_{st}/2$&$3n_{st}/4$&$n_{st}$&$T_{st}$ \\
    \hline
    $\|\cdot \|$& $8.696275e-02$ & $3.518164e-02$ & $1.821034e-02$ & $9.992221e-03$ &$429.507000$\\
\hline
$\triangle_{mass}$& $-1.634613e-02$ & $-1.638586e-02$ & $-1.639064e-02$ & $-1.639148e-02$ &\\
\hline
\end{tabular}

$\|v-v^{3}\|_{C_h} = 1.666309e-01$

\subsubsection{$\mu = 0.01, p(\rho) = 100\rho $}

\begin{tabular}{*{6}{|l}|}
    \hline
    \multicolumn{6}{|c|}{$ = 0.01, \tau = 0.0001$} \\
    \hline
    &$n_{st}/4 $&$ n_{st}/2$&$3n_{st}/4$&$n_{st}$&$T_{st}$ \\
    \hline
$\|\cdot \|$& $8.315025e-02$ & $3.644942e-02$ & $3.075703e-02$ & $9.999846e-03$ &$348.817500$\\
\hline
$\triangle_{mass}$& $-1.552336e-02$ & $-1.553799e-02$ & $-1.554157e-02$ & $-1.554311e-02$ &\\
\hline
\end{tabular}

$\|v-v^{2}\|_{C_h} = 2.229609e-01$
\subsubsection{$\mu = 0.01, p(\rho) = \rho^{1.4} $}

\begin{tabular}{*{6}{|l}|}
    \hline
    \multicolumn{6}{|c|}{$h = 0.01, \tau = 0.005$} \\
    \hline
    &$n_{st}/4 $&$ n_{st}/2$&$3n_{st}/4$&$n_{st}$&$T_{st}$ \\
    \hline
$\|\cdot \|$& $6.237204e-02$ & $3.652269e-02$ & $1.481286e-02$ & $9.986921e-03$ &$557.545000$\\
\hline
$\triangle_{mass}$& $-1.355200e-02$ & $-1.363429e-02$ & $-1.364490e-02$ & $-1.365908e-02$ &\\
\hline
\end{tabular}

$\|v-v^{4}\|_{C_h} = 1.450424e-01$


\begin{tabular}{*{6}{|l}|}
    \hline
    \multicolumn{6}{|c|}{$h = 0.001, \tau = 0.001$} \\
    \hline
    &$n_{st}/4 $&$ n_{st}/2$&$3n_{st}/4$&$n_{st}$&$T_{st}$ \\
    \hline
$\|\cdot \|$& $5.200368e-01$ & $2.256165e-01$ & $2.451412e-01$ & $9.996130e-02$ &$73.082000$\\
\hline
$\triangle_{mass}$& $-2.443758e-03$ & $-2.766124e-03$ & $-2.903238e-03$ & $-2.975294e-03$ &\\
\hline    
\end{tabular}

$\|v-v^{3}\|_{C_h} = 1.294856e-01$

\subsubsection{$\mu = 0.001, p(\rho) = 1\rho $}

\begin{tabular}{*{6}{|l}|}
    \hline
    \multicolumn{6}{|c|}{$h = 0.01, \tau = 0.001$} \\
    \hline
    &$n_{st}/4 $&$ n_{st}/2$&$3n_{st}/4$&$n_{st}$&$T_{st}$ \\
    \hline
    $\|\cdot \|$& $5.633435e-02$ & $2.351355e-02$ & $2.116416e-02$ & $9.993981e-03$ &$1015.715000$\\
\hline
$\triangle_{mass}$& $-1.539545e-02$ & $-1.551805e-02$ & $-1.551803e-02$ & $-1.553455e-02$ &\\
\hline
\end{tabular}

$\|v-v^{3}\|_{C_h} = 1.461955e-01$

\subsubsection{$\mu = 0.001, p(\rho) = 10\rho $}

\begin{tabular}{*{6}{|l}|}
    \hline
    \multicolumn{6}{|c|}{$h = 0.01, \tau = 0.001$} \\
    \hline
    &$n_{st}/4 $&$ n_{st}/2$&$3n_{st}/4$&$n_{st}$&$T_{st}$ \\
    \hline
$\|\cdot \|$& $2.486766e-01$ & $1.399602e-01$ & $9.206896e-02$ & $4.999601e-02$ &$181.045200$\\
\hline
$\triangle_{mass}$& $-1.470820e-02$ & $-1.506636e-02$ & $-1.513866e-02$ & $-1.513172e-02$ &\\
\hline
\end{tabular}

$\|v-v^{3}\|_{C_h} = 4.360167e-01$

\subsubsection{$\mu = 0.001, p(\rho) = 100\rho $}
Для этих параметров я не нашёл сетку, на которой стабилизировалось бы за достаточно короткое время.

\subsubsection{$\mu = 0.001, p(\rho) = \rho^{1.4} $}

\begin{tabular}{*{6}{|l}|}
    \hline
    \multicolumn{6}{|c|}{$h = , \tau = $} \\
    \hline
    &$n_{st}/4 $&$ n_{st}/2$&$3n_{st}/4$&$n_{st}$&$T_{st}$ \\
    \hline
$\|\cdot \|$& $3.577539e-01$ & $2.094300e-01$ & $1.369606e-01$ & $9.999421e-02$ &$113.967100$\\
\hline
$\triangle_{mass}$& $-2.691286e-03$ & $-3.084170e-03$ & $-3.071215e-03$ & $-3.171587e-03$ &\\
\hline
\end{tabular}

$\|v-v^{3}\|_{C_h} = 3.286338e-01$
\newpage
\subsubsection{Графики}
\begin{center}
\begin{figure}[H]
    \centering
    \includegraphics[height=0.4\textheight]{pics/task2/u-2-2-11_1.png}
    \caption{Проекция скорости для $\tau = 0.01, h = 0.01, \mu = 0.1, p(\rho) = \rho$}
\end{figure}

\begin{figure}[H]
    \centering
    \includegraphics[height=0.4\textheight]{pics/task2/u-2-2-12_1.png}
    \caption{Проекция скорости для $\tau = 0.01, h = 0.01, \mu = 0.1, p(\rho) = 10\rho$}
\end{figure}

\begin{figure}[H]
    \centering
    \includegraphics[height=0.4\textheight]{pics/task2/u-2-2-21_1.png}
    \caption{Проекция скорости для $\tau = 0.01, h = 0.01, \mu = 0.01, p(\rho) = \rho$}
\end{figure}

\begin{figure}[H]
    \centering
    \includegraphics[height=0.4\textheight]{pics/task2/h-2-2-11_1.png}
    \caption{Проекция плотности для $\tau = 0.01, h = 0.01, \mu = 0.1, p(\rho) = \rho$}
\end{figure}

\begin{figure}[H]
    \centering
    \includegraphics[height=0.4\textheight]{pics/task2/h-2-2-12_1.png}
    \caption{Проекция плотности для $\tau = 0.01, h = 0.01, \mu = 0.1, p(\rho) = 10\rho$}
\end{figure}

\begin{figure}[H]
    \centering
    \includegraphics[height=0.4\textheight]{pics/task2/h-2-2-21_1.png}
    \caption{Проекция плотности для $\tau = 0.01, h = 0.01, \mu = 0.01, p(\rho) = \rho$}
\end{figure}
\end{center}

На графиках с одинаковой зависимостью $p$ от $\rho$ видно, что длина цикла примерно одинакова и лежит между 9 и 10, но при этом при меньшей вязкости стабилизация происходит медленнее. При большем коэффициенте зависимости длина цикла уменьшилась.

\begin{figure}[h]
	\begin{minipage}[h]{0.47\linewidth}
		\centering
		\includegraphics[width=1\linewidth]{pics/task2/14h_1.png} 
		\caption{Плотность на слое $n_{st} / 4$}
	\end{minipage}
	\hfill
	\begin{minipage}[h]{0.47\linewidth}
		\centering
		\includegraphics[width=1\linewidth]{pics/task2/24h_1.png} 
		\caption{Плотность на слое $n_{st} / 2$}
	\end{minipage}
	\vfill
	\begin{minipage}[h]{0.47\linewidth}
		\centering
		\includegraphics[width=1\linewidth]{pics/task2/34h_1.png} 
		\caption{Плотность на слое $3n_{st} / 4$}
	\end{minipage}
	\hfill
	\begin{minipage}[h]{0.47\linewidth}
		\centering
		\includegraphics[width=1\linewidth]{pics/task2/44h_1.png} 
		\caption{Плотность на слое $n_{st}$}
	\end{minipage}
	\caption{Графики плотности для $\tau = 0.01, h = 0.01, \mu = 0.1, p(\rho) = \rho$}
\end{figure}

\begin{figure}[h]
	\begin{minipage}[h]{0.47\linewidth}
		\centering
		\includegraphics[width=1\linewidth]{pics/task2/14u_1.png} 
		\caption{Скорость на слое $n_{st} / 4$}
	\end{minipage}
	\hfill
	\begin{minipage}[h]{0.47\linewidth}
		\centering
		\includegraphics[width=1\linewidth]{pics/task2/24u_1.png} 
		\caption{Скорость на слое $n_{st} / 2$}
	\end{minipage}
	\vfill
	\begin{minipage}[h]{0.47\linewidth}
		\centering
		\includegraphics[width=1\linewidth]{pics/task2/34u_1.png} 
		\caption{Скорость на слое $3n_{st} / 4$}
	\end{minipage}
	\hfill
	\begin{minipage}[h]{0.47\linewidth}
		\centering
		\includegraphics[width=1\linewidth]{pics/task2/44u_1.png} 
		\caption{Скорость на слое $n_{st}$}
	\end{minipage}
	\caption{Графики скорости для $\tau = 0.01, h = 0.01, \mu = 0.1, p(\rho) = \rho$}
\end{figure}

\subsection{Вторая задача}
\subsubsection{$\mu = 0.1, p(\rho) = \rho $}
\begin{tabular}{*{6}{|l}|}
    \hline
    \multicolumn{6}{|c|}{$h = 0.1, \tau = 0.1$} \\
    \hline
    &$n_{st}/4 $&$ n_{st}/2$&$3n_{st}/4$&$n_{st}$&$T_{st}$ \\
    \hline
    $\|\cdot \|$& $1.057994e-01$ & $4.497699e-02$ & $2.105080e-02$ & $9.937196e-03$ &$475.300000$\\
\hline
$\triangle_{mass}$& $-2.151764e-02$ & $-2.169077e-02$ & $-2.170753e-02$ & $-2.171401e-02$ &\\
\hline
\end{tabular}

$\|v-v^{4}\|_{C_h} = 6.210459e-02$


\begin{tabular}{*{6}{|l}|}
    \hline
    \multicolumn{6}{|c|}{$h = 0.01, \tau = 0.01$} \\
    \hline
    &$n_{st}/4 $&$ n_{st}/2$&$3n_{st}/4$&$n_{st}$&$T_{st}$ \\
    \hline
    $\|\cdot \|$& $8.115221e-02$ & $3.328051e-02$ & $1.690131e-02$ & $9.974448e-03$ &$439.410000$\\
\hline
$\triangle_{mass}$& $-2.523587e-03$ & $-2.565776e-03$ & $-2.574194e-03$ & $-2.576565e-03$ &\\
\hline
\end{tabular}

$\|v-v^{4}\|_{C_h} = 7.383419e-03$

\begin{tabular}{*{6}{|l}|}
    \hline
    \multicolumn{6}{|c|}{$h = 0.01, \tau = 0.001$} \\
    \hline
    &$n_{st}/4 $&$ n_{st}/2$&$3n_{st}/4$&$n_{st}$&$T_{st}$ \\
    \hline
    $\|\cdot \|$& $8.448036e-02$ & $3.432835e-02$ & $1.725112e-02$ & $9.996768e-03$ &$439.318000$\\
\hline
$\triangle_{mass}$& $-2.568542e-04$ & $-2.650773e-04$ & $-2.666425e-04$ & $-2.669819e-04$ &\\
\hline
\end{tabular}

$\|v-v^{3}\|_{C_h} = 1.323868e-03$

\subsubsection{$\mu = 0.1, p(\rho) = 10\rho $}

\begin{tabular}{*{6}{|l}|}
    \hline
    \multicolumn{6}{|c|}{$ = 0.1, \tau = 0.01$} \\
    \hline
    &$n_{st}/4 $&$ n_{st}/2$&$3n_{st}/4$&$n_{st}$&$T_{st}$ \\
    \hline
$\|\cdot \|$& $1.298147e-01$ & $5.389401e-02$ & $2.525908e-02$ & $9.947040e-03$ &$289.360000$\\
\hline
$\triangle_{mass}$& $-2.602626e-03$ & $-2.598529e-03$ & $-2.582382e-03$ & $-2.574958e-03$ &\\
\hline
\end{tabular}

$\|v-v^{4}\|_{C_h} = 2.998523e-02$

\begin{tabular}{*{6}{|l}|}
    \hline
    \multicolumn{6}{|c|}{$h = 0.01, \tau = 0.01$} \\
    \hline
    &$n_{st}/4 $&$ n_{st}/2$&$3n_{st}/4$&$n_{st}$&$T_{st}$ \\
    \hline
    $\|\cdot \|$& $8.459588e-02$ & $5.396960e-02$ & $3.536468e-02$ & $9.987376e-03$ &$286.210000$\\
\hline
$\triangle_{mass}$& $-2.345968e-03$ & $-2.415463e-03$ & $-2.440618e-03$ & $-2.445535e-03$ &\\
\hline
\end{tabular}

$\|v-v^{4}\|_{C_h} = 5.722661e-03$

\begin{tabular}{*{6}{|l}|}
    \hline
    \multicolumn{6}{|c|}{$h = 0.01, \tau = 0.001$} \\
    \hline
    &$n_{st}/4 $&$ n_{st}/2$&$3n_{st}/4$&$n_{st}$&$T_{st}$ \\
    \hline
    $\|\cdot \|$& $8.780402e-02$ & $5.311081e-02$ & $3.559624e-02$ & $9.992142e-03$ &$286.119000$\\
\hline
$\triangle_{mass}$& $-2.475217e-04$ & $-2.437268e-04$ & $-2.524342e-04$ & $-2.513063e-04$ &\\
\hline
\end{tabular}
$\|v-v^{3}\|_{C_h} = 2.622959e-03$

\subsubsection{$\mu = 0.1, p(\rho) = 100\rho $}

\begin{tabular}{*{6}{|l}|}
    \hline
    \multicolumn{6}{|c|}{$h = 0.01, \tau = 0.001$} \\
    \hline
    &$n_{st}/4 $&$ n_{st}/2$&$3n_{st}/4$&$n_{st}$&$T_{st}$ \\
    \hline
    $\|\cdot \|$& $8.653968e-02$ & $5.630521e-02$ & $4.426884e-02$ & $9.966610e-03$ &$217.502000$\\
\hline
$\triangle_{mass}$& $-2.296218e-04$ & $-2.474407e-04$ & $-2.469809e-04$ & $-2.491424e-04$ &\\
\hline
\end{tabular}

$\|v-v^{4}\|_{C_h} = 4.344025e-03$

\begin{tabular}{*{6}{|l}|}
    \hline
    \multicolumn{6}{|c|}{$h = 0.001, \tau = 0.001$} \\
    \hline
    &$n_{st}/4 $&$ n_{st}/2$&$3n_{st}/4$&$n_{st}$&$T_{st}$ \\
    \hline
    $\|\cdot \|$& $1.351982e-01$ & $5.697122e-02$ & $2.724762e-02$ & $9.986078e-03$ &$216.503000$\\
\hline
$\triangle_{mass}$& $-2.303206e-04$ & $-2.431240e-04$ & $-2.462610e-04$ & $-2.474967e-04$ &\\
\hline
\end{tabular}

$\|v-v^{2}\|_{C_h} = 2.747033e-03$

\subsubsection{$\mu = 0.1, p(\rho) = \rho^{1.4} $}
\begin{tabular}{*{6}{|l}|}
    \hline
    \multicolumn{6}{|c|}{$h = 0.1, \tau = 0.1$} \\
    \hline
    &$n_{st}/4 $&$ n_{st}/2$&$3n_{st}/4$&$n_{st}$&$T_{st}$ \\
    \hline
    $\|\cdot \|$& $8.849004e-02$ & $4.460481e-02$ & $2.274127e-02$ & $9.816406e-03$ &$386.600000$\\
\hline
$\triangle_{mass}$& $-2.142117e-02$ & $-2.194042e-02$ & $-2.194306e-02$ & $-2.197424e-02$ &\\
\hline
\end{tabular}

$\|v-v^{4}\|_{C_h} = 1.089634e-01$

\begin{tabular}{*{6}{|l}|}
    \hline
    \multicolumn{6}{|c|}{$h = 0.01, \tau = 0.01$} \\
    \hline
    &$n_{st}/4 $&$ n_{st}/2$&$3n_{st}/4$&$n_{st}$&$T_{st}$ \\
    \hline
    $\|\cdot \|$& $8.177089e-02$ & $4.732491e-02$ & $2.412279e-02$ & $9.999100e-03$ &$384.230000$\\
\hline
$\triangle_{mass}$& $-2.475776e-03$ & $-2.528510e-03$ & $-2.528038e-03$ & $-2.532089e-03$ &\\
\hline
\end{tabular}

$\|v-v^{4}\|_{C_h} = 1.319770e-02$

\begin{tabular}{*{6}{|l}|}
    \hline
    \multicolumn{6}{|c|}{$h = 0.01, \tau = 0.001$} \\
    \hline
    &$n_{st}/4 $&$ n_{st}/2$&$3n_{st}/4$&$n_{st}$&$T_{st}$ \\
    \hline
    $\|\cdot \|$& $8.999629e-02$ & $4.564517e-02$ & $2.333583e-02$ & $9.999805e-03$ &$392.287000$\\
\hline
$\triangle_{mass}$& $-2.703800e-04$ & $-2.551330e-04$ & $-2.638645e-04$ & $-2.623066e-04$ &\\
\hline
\end{tabular}

$\|v-v^{3}\|_{C_h} = 1.200642e-03$

\begin{tabular}{*{6}{|l}|}
    \hline
    \multicolumn{6}{|c|}{$h = 0.001, \tau = 0.01$} \\
    \hline
    &$n_{st}/4 $&$ n_{st}/2$&$3n_{st}/4$&$n_{st}$&$T_{st}$ \\
    \hline
$\|\cdot \|$& $8.171681e-02$ & $4.723875e-02$ & $2.407144e-02$ & $9.998224e-03$ &$384.230000$\\
\hline
$\triangle_{mass}$& $-2.463330e-03$ & $-2.500851e-03$ & $-2.507733e-03$ & $-2.510269e-03$ &\\
\hline  
\end{tabular}

$\|v-v^{3}\|_{C_h} = 1.281673e-02$

\subsubsection{$\mu = 0.01, p(\rho) = \rho $}

\begin{tabular}{*{6}{|l}|}
    \hline
    \multicolumn{6}{|c|}{$h = 0.01, \tau = 0.01$} \\
    \hline
    &$n_{st}/4 $&$ n_{st}/2$&$3n_{st}/4$&$n_{st}$&$T_{st}$ \\
    \hline
$\|\cdot \|$& $3.755305e-02$ & $2.563334e-02$ & $2.026205e-02$ & $9.988237e-03$ &$1147.100000$\\
\hline
$\triangle_{mass}$& $-2.257917e-02$ & $-2.262130e-02$ & $-2.264216e-02$ & $-2.264381e-02$ &\\
\hline
\end{tabular}

$\|v-v^{4}\|_{C_h} = 1.498957e-01$

\begin{tabular}{*{6}{|l}|}
    \hline
    \multicolumn{6}{|c|}{$h = 0.01, \tau = 0.001$} \\
    \hline
    &$n_{st}/4 $&$ n_{st}/2$&$3n_{st}/4$&$n_{st}$&$T_{st}$ \\
    \hline
    $\|\cdot \|$& $4.756873e-02$ & $2.231827e-02$ & $1.895506e-02$ & $9.999852e-03$ &$1187.069000$\\
\hline
$\triangle_{mass}$& $-2.737012e-03$ & $-2.731282e-03$ & $-2.728745e-03$ & $-2.733232e-03$ &\\
\hline
\end{tabular}

$\|v-v^{3}\|_{C_h} = 3.385155e-02$

\begin{tabular}{*{6}{|l}|}
    \hline
    \multicolumn{6}{|c|}{$h = , \tau = $} \\
    \hline
    &$n_{st}/4 $&$ n_{st}/2$&$3n_{st}/4$&$n_{st}$&$T_{st}$ \\
    \hline
    $\|\cdot \|$& $4.624345e-02$ & $2.465977e-02$ & $1.436130e-02$ & $9.993862e-03$ &$1242.180000$\\
\hline
$\triangle_{mass}$& $-2.249926e-02$ & $-2.254468e-02$ & $-2.255458e-02$ & $-2.255879e-02$ &\\
\hline
\end{tabular}

$\|v-v^{2}\|_{C_h} = 1.291567e-01$

\subsubsection{$\mu = 0.01, p(\rho) = 10\rho $}

\begin{tabular}{*{6}{|l}|}
    \hline
    \multicolumn{6}{|c|}{$h = 0.1, \tau = 0.001$} \\
    \hline
    &$n_{st}/4 $&$ n_{st}/2$&$3n_{st}/4$&$n_{st}$&$T_{st}$ \\
    \hline
$\|\cdot \|$& $6.888265e-02$ & $3.459087e-02$ & $1.825954e-02$ & $9.997479e-03$ &$693.024000$\\
\hline
$\triangle_{mass}$& $-2.631510e-03$ & $-2.570296e-03$ & $-2.565111e-03$ & $-2.545413e-03$ &\\
\hline

\end{tabular}

$\|v-v^{4}\|_{C_h} = 1.399669e-01$

\begin{tabular}{*{6}{|l}|}
    \hline
    \multicolumn{6}{|c|}{$h = , \tau = $} \\
    \hline
    &$n_{st}/4 $&$ n_{st}/2$&$3n_{st}/4$&$n_{st}$&$T_{st}$ \\
    \hline
    $\|\cdot \|$& $4.596806e-02$ & $2.721373e-02$ & $2.582281e-02$ & $9.997545e-03$ &$577.493000$\\
\hline
$\triangle_{mass}$& $-2.456556e-03$ & $-2.470025e-03$ & $-2.479802e-03$ & $-2.479199e-03$ &\\
\hline
\end{tabular}

 $\|v-v^{3}\|_{C_h} = 1.267514e-01$
\subsubsection{$\mu = 0.01, p(\rho) = 100\rho $}

\begin{tabular}{*{6}{|l}|}
    \hline
    \multicolumn{6}{|c|}{$h = 0.01, \tau = 0.0001$} \\
    \hline
    &$n_{st}/4 $&$ n_{st}/2$&$3n_{st}/4$&$n_{st}$&$T_{st}$ \\
    \hline
$\|\cdot \|$& $7.214288e-02$ & $2.692124e-02$ & $1.536680e-02$ & $9.998859e-03$ &$480.526600$\\
\hline
$\triangle_{mass}$& $-2.426798e-04$ & $-2.481020e-04$ & $-2.495873e-04$ & $-2.502651e-04$ &\\
\hline
\end{tabular}

$\|v-v^{2}\|_{C_h} = 5.224555e-02$

\subsubsection{$\mu = 0.01, p(\rho) = \rho^{1.4} $}

\begin{tabular}{*{6}{|l}|}
    \hline
    \multicolumn{6}{|c|}{$h = 0.1, \tau = 0.01$} \\
    \hline
    &$n_{st}/4 $&$ n_{st}/2$&$3n_{st}/4$&$n_{st}$&$T_{st}$ \\
    \hline$\|\cdot \|$& $6.206453e-02$ & $2.635210e-02$ & $1.491993e-02$ & $9.996229e-03$ &$948.120000$\\
\hline
$\triangle_{mass}$& $-2.140423e-02$ & $-2.139353e-02$ & $-2.137896e-02$ & $-2.136954e-02$ &\\
\hline
\end{tabular}
$\|v-v^{4}\|_{C_h} = 1.350754e-01$

\begin{tabular}{*{6}{|l}|}
    \hline
    \multicolumn{6}{|c|}{$h = 0.01, \tau = 0.001$} \\
    \hline
    &$n_{st}/4 $&$ n_{st}/2$&$3n_{st}/4$&$n_{st}$&$T_{st}$ \\
    \hline$\|\cdot \|$& $3.638901e-02$ & $2.530445e-02$ & $2.101946e-02$ & $9.997771e-03$ &$990.525000$\\
\hline
$\triangle_{mass}$& $-2.634875e-03$ & $-2.648825e-03$ & $-2.642002e-03$ & $-2.645937e-03$ &\\
\hline
\end{tabular}

$\|v-v^{2}\|_{C_h} = 4.821490e-02$

\subsubsection{$\mu = 0.001, p(\rho) = \rho $}

\begin{tabular}{*{6}{|l}|}
    \hline
    \multicolumn{6}{|c|}{$h = 0.01, \tau = 0.001$} \\
    \hline
    &$n_{st}/4 $&$ n_{st}/2$&$3n_{st}/4$&$n_{st}$&$T_{st}$ \\
    \hline$\|\cdot \|$& $3.293496e-02$ & $1.889809e-02$ & $1.569922e-02$ & $9.999973e-03$ &$1849.610000$\\
\hline
$\triangle_{mass}$& $-2.126218e-02$ & $-2.128896e-02$ & $-2.129372e-02$ & $-2.129839e-02$ &\\
\hline
\end{tabular}

$\|v-v^{3}\|_{C_h} = 9.548446e-02$

\begin{tabular}{*{6}{|l}|}
    \hline
    \multicolumn{6}{|c|}{$h = 0.001, \tau = 0.0001$} \\
    \hline
    &$n_{st}/4 $&$ n_{st}/2$&$3n_{st}/4$&$n_{st}$&$T_{st}$ \\
    \hline
$\|\cdot \|$& $7.943369e-02$ & $3.416127e-02$ & $2.313465e-02$ & $1.861502e-02$ &$1000.000000$\\
\hline
$\triangle_{mass}$& $-3.133873e-03$ & $-3.152216e-03$ & $-3.156661e-03$ & $-3.158007e-03$ &\\
\hline   
   \end{tabular}

$\|v-v^{1}\|_{C_h} = 1.189583e-01$

\subsubsection{$\mu = 0.001, p(\rho) = 10\rho $}

\begin{tabular}{*{6}{|l}|}
    \hline
    \multicolumn{6}{|c|}{$h = 0.01, \tau = 0.0001$} \\
    \hline
    &$n_{st}/4 $&$ n_{st}/2$&$3n_{st}/4$&$n_{st}$&$T_{st}$ \\
    \hline
$\|\cdot \|$& $5.654326e-02$ & $2.418228e-02$ & $1.529745e-02$ & $9.999739e-03$ &$748.537500$\\
\hline
$\triangle_{mass}$& $-2.407347e-03$ & $-2.424248e-03$ & $-2.428702e-03$ & $-2.430007e-03$ &\\
\hline
\end{tabular}

$\|v-v^{2}\|_{C_h} = 1.636180e-01$

\subsubsection{$\mu = 0.001, p(\rho) = 100\rho $}

Я не подобрал сетку, на которой вычисления происходили бы достаточно быстро и не расходились.

\subsubsection{$\mu = 0.001, p(\rho) = \rho^{1.4} $}
\begin{tabular}{*{6}{|l}|}
    \hline
    \multicolumn{6}{|c|}{$h = 0.01, \tau = 0.0001$} \\
    \hline
    &$n_{st}/4 $&$ n_{st}/2$&$3n_{st}/4$&$n_{st}$&$T_{st}$ \\
    \hline
    $\|\cdot \|$& $6.623326e-01$ & $3.439675e-01$ & $2.141368e-01$ & $1.480272e-01$ &$100.000000$\\
\hline
$\triangle_{mass}$& $-2.337562e-03$ & $-2.729491e-03$ & $-2.792877e-03$ & $-2.810747e-03$ &\\
\hline
\end{tabular}

$\|v-v^{2}\|_{C_h} = 3.783958e-01$

\subsubsection{Графики}
\begin{center}
\begin{figure}[H]
    \centering
    \includegraphics[height=0.4\textheight]{pics/task2/u-2-2-11_2.png}
    \caption{Проекция скорости для $\tau = 0.01, h = 0.01, \mu = 0.1, p(\rho) = \rho$}
\end{figure}

\begin{figure}[H]
    \centering
    \includegraphics[height=0.4\textheight]{pics/task2/u-2-2-12_2.png}
    \caption{Проекция скорости для $\tau = 0.01, h = 0.01, \mu = 0.1, p(\rho) = 10\rho$}
\end{figure}

\begin{figure}[H]
    \centering
    \includegraphics[height=0.4\textheight]{pics/task2/u-2-2-21_2.png}
    \caption{Проекция скорости для $\tau = 0.01, h = 0.01, \mu = 0.01, p(\rho) = \rho$}
\end{figure}

\begin{figure}[H]
    \centering
    \includegraphics[height=0.4\textheight]{pics/task2/h-2-2-11_2.png}
    \caption{Проекция плотности для $\tau = 0.01, h = 0.01, \mu = 0.1, p(\rho) = \rho$}
\end{figure}

\begin{figure}[H]
    \centering
    \includegraphics[height=0.4\textheight]{pics/task2/h-2-2-12_2.png}
    \caption{Проекция плотности для $\tau = 0.01, h = 0.01, \mu = 0.1, p(\rho) = 10\rho$}
\end{figure}

\begin{figure}[H]
    \centering
    \includegraphics[height=0.4\textheight]{pics/task2/h-2-2-21_2.png}
    \caption{Проекция плотности для $\tau = 0.01, h = 0.01, \mu = 0.01, p(\rho) = \rho$}
\end{figure}
\end{center}


\begin{figure}[h]
	\begin{minipage}[h]{0.47\linewidth}
		\centering
		\includegraphics[width=1\linewidth]{pics/task2/14h_2.png} 
		\caption{Плотность на слое $n_{st} / 4$}
	\end{minipage}
	\hfill
	\begin{minipage}[h]{0.47\linewidth}
		\centering
		\includegraphics[width=1\linewidth]{pics/task2/24h_2.png} 
		\caption{Плотность на слое $n_{st} / 2$}
	\end{minipage}
	\vfill
	\begin{minipage}[h]{0.47\linewidth}
		\centering
		\includegraphics[width=1\linewidth]{pics/task2/34h_2.png} 
		\caption{Плотность на слое $3n_{st} / 4$}
	\end{minipage}
	\hfill
	\begin{minipage}[h]{0.47\linewidth}
		\centering
		\includegraphics[width=1\linewidth]{pics/task2/44h_2.png} 
		\caption{Плотность на слое $n_{st}$}
	\end{minipage}
	\caption{Графики плотности для $\tau = 0.01, h = 0.01, \mu = 0.1, p(\rho) = \rho$}
\end{figure}

\begin{figure}[h]
	\begin{minipage}[h]{0.47\linewidth}
		\centering
		\includegraphics[width=1\linewidth]{pics/task2/14u_2.png} 
		\caption{Скорость на слое $n_{st} / 4$}
	\end{minipage}
	\hfill
	\begin{minipage}[h]{0.47\linewidth}
		\centering
		\includegraphics[width=1\linewidth]{pics/task2/24u_2.png} 
		\caption{Скорость на слое $n_{st} / 2$}
	\end{minipage}
	\vfill
	\begin{minipage}[h]{0.47\linewidth}
		\centering
		\includegraphics[width=1\linewidth]{pics/task2/34u_2.png} 
		\caption{Скорость на слое $3n_{st} / 4$}
	\end{minipage}
	\hfill
	\begin{minipage}[h]{0.47\linewidth}
		\centering
		\includegraphics[width=1\linewidth]{pics/task2/44u_2.png} 
		\caption{Скорость на слое $n_{st}$}
	\end{minipage}
	\caption{Графики скорости для $\tau = 0.01, h = 0.01, \mu = 0.1, p(\rho) = \rho$}
\end{figure}




\section{Стабилизация осцилирующей функции}

\begin{landscape} 
\begin{tabular}{|c|c|c|c|c|c|c|c|c|c|c|c|c|c|c|c}
\hline
$k$          & $\begin{array}{c}\mu = 0.1\\p(\rho) = \rho\\\tau = 0.0001\\h = 0.001\end{array}$ & $\begin{array}{c}\mu = 0.1\\p(\rho) = \rho\\\tau = 0.0001\\h = 0.001\end{array}$ & $\begin{array}{c}\mu = 0.1\\p(\rho) = 10\rho\\\tau = 0.0001\\h = 0.001\end{array}$ & $\begin{array}{c}\mu = 0.1\\p(\rho) = 100\rho\\\tau = 0.0001\\h = 0.001\end{array}$ & $\begin{array}{c}\mu = 0.1\\p(\rho) = \rho^{1,4}\\\tau = 0.0001\\h = 0.001\end{array}$ & $\begin{array}{c}\mu = 0.01\\p(\rho) = \rho\\\tau = 0.0001\\h = 0.001\end{array}$ & $\begin{array}{c}\mu = 0.01\\p(\rho) = 10\rho\\\tau = 0.0001\\h = 0.001\end{array}$ & $\begin{array}{c}\mu = 0.01\\p(\rho) = 100\rho\\\tau = 0.0001\\h = 0.001\end{array}$ & $\begin{array}{c}\mu = 0.01\\p(\rho) = \rho^{1,4}\\\tau = 0.0001\\h = 0.001\end{array}$ & $\begin{array}{c}\mu = 0.001\\p(\rho) = \rho\\\tau = 0.0001\\h = 0.001\end{array}$ & $\begin{array}{c}\mu = 0.001\\p(\rho) = 10\rho\\\tau = 0.0001\\h = 0.001\end{array}$ & $\begin{array}{c}\mu = 0.001\\p(\rho) = \rho\\\tau = 0.00001\\h = 0.001\end{array}$ & $\begin{array}{c}\mu = 0.001\\p(\rho) = 10\rho\\\tau = 0.00001\\h = 0.001\end{array}$ & $\begin{array}{c}\mu = 0.001\\p(\rho) = 100\rho\\\tau = 0.00001\\h = 0.001\end{array}$ & \multicolumn{1}{c|}{$\begin{array}{c}\mu = 0.001\\p(\rho) = \rho^{1,4}\\\tau = 0.00001\\h = 0.001\end{array}$} \\ \hline
$1$          & $9.0075$                                                        & $13.5300$                                                       & $11.2271$                                                         & $9.0501$                                                           & $13.0990$                                                             & $97.3549$                                                       & $75.4066$                                                         & $59.1501$                                                          & $90.7640$                                                             & $1000.0000$                                                     & $360.9727$                                                        & $1000.0000$                                                     & $364.43658$                                                        & $292.24996$                                                         & \multicolumn{1}{c|}{$1000.0000$}                                                           \\ \hline
$2$          & $2.2926$                                                        & $3.6143$                                                        & $2.9313$                                                          & $2.3258$                                                           & $3.4785$                                                              & $27.6879$                                                       & $22.0512$                                                         & $17.3752$                                                          & $26.3597$                                                             & $1000.0000$                                                     & $117.0830$                                                        & $1000.0000$                                                     & $118.34062$                                                        & $93.9752$                                                          & \multicolumn{1}{c|}{$185.28905$}                                                            \\ \hline
$3$          & $1.0952$                                                        & $1.5778$                                                        & $1.3250$                                                          & $1.0508$                                                           & $1.5655$                                                              & $13.1324$                                                       & $10.5907$                                                         & $8.3836$                                                           & $12.5078$                                                             & $376.2377$                                                      & $59.5036$                                                         & $253.76046$                                                      & $60.13129$                                                         & $47.48365$                                                          & \multicolumn{1}{c|}{$84.92186$}                                                             \\ \hline
$4$          & $0.6191$                                                        & $0.9295$                                                        & $0.7588$                                                          & $0.5884$                                                           & $0.9434$                                                              & $7.7336$                                                        & $6.2833$                                                          & $4.9878$                                                           & $7.4763$                                                              & $99.8299$                                                       & $36.5641$                                                         & $83.75953$                                                       & $36.87686$                                                         & $29.0629$                                                          & \multicolumn{1}{c|}{$50.16533$}                                                             \\ \hline
$5$          & $0.4547$                                                        & $0.6173$                                                        & $0.4831$                                                          & $0.3909$                                                           & $0.6195$                                                              & $5.0868$                                                        & $4.2045$                                                          & $3.3503$                                                           & $4.8088$                                                              & $59.0631$                                                       & $24.9509$                                                         & $51.05168$                                                       & $25.20103$                                                         & $19.79046$                                                          & \multicolumn{1}{c|}{$34.03985$}                                                             \\ \hline
$6$          & $0.3610$                                                        & $0.4804$                                                        & $0.3504$                                                          & $0.2759$                                                           & $0.4350$                                                              & $3.6600$                                                        & $2.9773$                                                          & $2.3920$                                                           & $3.4447$                                                              & $34.1694$                                                       & $18.2101$                                                         & $34.17169$                                                       & $18.41857$                                                         & $14.42551$                                                          & \multicolumn{1}{c|}{$24.28543$}                                                             \\ \hline
$7$          & $0.3748$                                                        & $0.5460$                                                        & $0.2570$                                                          & $0.1938$                                                           & $0.3421$                                                              & $2.7139$                                                        & $2.2360$                                                          & $1.7932$                                                           & $2.5908$                                                              & $24.1403$                                                       & $13.9825$                                                         & $24.14343$                                                       & $14.11601$                                                         & $11.03627$                                                          & \multicolumn{1}{c|}{$18.40213$}                                                             \\ \hline
$8$          & $0.3753$                                                        & $0.5692$                                                        & $0.1879$                                                          & $0.1572$                                                           & $0.3848$                                                              & $2.1253$                                                        & $1.7198$                                                          & $1.4066$                                                           & $2.0527$                                                              & $18.1200$                                                       & $11.0885$                                                         & $18.12225$                                                       & $11.16584$                                                         & $8.73184$                                                           & \multicolumn{1}{c|}{$14.51292$}                                                             \\ \hline
$9$          & $0.3691$                                                        & $0.5683$                                                        & $0.1657$                                                          & $0.1287$                                                           & $0.3947$                                                              & $1.6726$                                                        & $1.3882$                                                          & $1.1171$                                                           & $1.6378$                                                              & $14.1104$                                                       & $9.0134$                                                          & $14.11191$                                                       & $9.11717$                                                          & $7.10618$                                                           & \multicolumn{1}{c|}{$11.68362$}                                                             \\ \hline
$10$         & $0.3628$                                                        & $0.5754$                                                        & $0.1213$                                                          & $0.0960$                                                           & $0.4044$                                                              & $1.3545$                                                        & $1.1233$                                                          & $0.9054$                                                           & $1.3113$                                                              & $11.1377$                                                       & $7.4798$                                                          & $11.58878$                                                       & $7.57318$                                                          & $5.89565$                                                           & \multicolumn{1}{c|}{$9.41665$}                                                              \\ \hline
$10 + M/10$  & $0.1635$                                                        & $0.7556$                                                        & $0.0837$                                                          & $0.0168$                                                           & $0.5449$                                                              & $0.0764$                                                        & $0.0180$                                                          & $0.0204$                                                           & $0.0589$                                                              & $0.1319$                                                        & $0.1015$                                                          & $0.13146$                                                        & $0.10117$                                                          & $0.0793$                                                           & \multicolumn{1}{c|}{$0.1289$}                                                              \\ \hline
$10 + 2M/10$ & $0.1825$                                                        & $2.8300$                                                        & $0.2894$                                                          & $0.0642$                                                           & $2.0303$                                                              & $0.3134$                                                        & $0.0636$                                                          & $0.0846$                                                           & $0.2333$                                                              & $0.1408$                                                        & $0.0764$                                                          & $0.11901$                                                        & $0.03666$                                                          & $0.02592$                                                           & \multicolumn{1}{c|}{$0.1014$}                                                              \\ \hline
$10 + 3M/10$ & $0.2578$                                                        & $11.5012$                                                       & $1.2011$                                                          & $0.1910$                                                           & $8.2102$                                                              & $1.3228$                                                        & $0.2554$                                                          & $0.0988$                                                           & $0.9799$                                                              & $0.3828$                                                        & $0.3152$                                                          & $0.16866$                                                        & $0.11348$                                                          & $0.04943$                                                           & \multicolumn{1}{c|}{$0.11242$}                                                              \\ \hline
$10 + 4M/10$ & $0.6782$                                                        & $50.0426$                                                       & $5.0295$                                                          & $0.5933$                                                           & $35.7385$                                                             & $5.6346$                                                        & $0.7788$                                                          & $0.2985$                                                           & $4.1069$                                                              & $0.7851$                                                        & $1.1070$                                                          & $0.38727$                                                        & $0.15959$                                                          & $0.04999$                                                           & \multicolumn{1}{c|}{$0.2913$}                                                              \\ \hline
$10 + 5M/10$ & $1.9042$                                                        & $175.5118$                                                      & $17.5965$                                                         & $1.8867$                                                           & $125.4128$                                                            & $18.4547$                                                       & $2.6678$                                                          & $0.6985$                                                           & $13.6777$                                                             & $4.1991$                                                        & $2.3710$                                                          & $1.28788$                                                        & $0.22248$                                                          & $0.09994$                                                           & \multicolumn{1}{c|}{$0.91002$}                                                              \\ \hline
$10 + 6M/10$ & $4.3580$                                                        & $420.9375$                                                      & $42.2108$                                                         & $4.3732$                                                           & $300.7061$                                                            & $45.4542$                                                       & $5.3607$                                                          & $1.6485$                                                           & $32.5163$                                                             & $10.5327$                                                       & $3.6324$                                                          & $1.20113$                                                        & $0.47511$                                                          & $0.15287$                                                           & \multicolumn{1}{c|}{$0.0008$}                                                              \\ \hline
$10 + 7M/10$ & $6.2966$                                                        & $615.7032$                                                      & $61.7642$                                                         & $6.3582$                                                           & $439.7957$                                                            & $67.4558$                                                       & $7.8830$                                                          & $2.1481$                                                           & $48.0950$                                                             & $18.5646$                                                       & $3.9386$                                                          & $1.93877$                                                        & $0.47694$                                                          & $0.1530$                                                           & \multicolumn{1}{c|}{$0.00093$}                                                              \\ \hline
$10 + 8M/10$ & $4.8590$                                                        & $472.5872$                                                      & $47.3930$                                                         & $4.8659$                                                           & $337.5925$                                                            & $51.4150$                                                       & $6.1487$                                                          & $1.4483$                                                           & $36.8062$                                                             & $13.3006$                                                       & $2.9984$                                                          & $1.4253$                                                        & $0.79462$                                                          & $0.24765$                                                           & \multicolumn{1}{c|}{$1.29567$}                                                              \\ \hline
$10 + 9M/10$ & $1.6771$                                                        & $128.6127$                                                      & $12.8898$                                                         & $1.3923$                                                           & $91.8983$                                                             & $13.8856$                                                       & $1.7250$                                                          & $0.4481$                                                           & $10.2870$                                                             & $4.4118$                                                        & $2.0468$                                                          & $1.11841$                                                        & $1.1269$                                                          & $0.84543$                                                           & $1.00813$                                                                                   \\ \hline 
\end{tabular}
\end{landscape}


\hline
$\rho\\u$& 1& 2& 3& 4& 5& 6& 7& 8& 9\\
\hline
& 1&$46.728$&$26.974$&$13.346$&$9.084$&$7.451$&$6.827$&$5.886$&$5.034$&$4.422$\\
\hline
& 2&$32.229$&$16.128$&$10.923$&$8.399$&$6.915$&$6.342$&$5.391$&$4.769$&$4.199$\\
\hline
& 3&$18.689$&$13.298$&$10.092$&$7.938$&$6.517$&$6.079$&$5.319$&$4.660$&$4.132$\\
\hline
& 4&$84.115$&$12.131$&$9.707$&$8.088$&$7.217$&$5.946$&$5.282$&$4.603$&$4.079$\\
\hline
& 5&$95.638$&$11.455$&$9.586$&$7.893$&$7.017$&$6.026$&$5.204$&$4.549$&$4.048$\\
\hline
& 6&$105.071$&$11.135$&$9.374$&$7.724$&$6.941$&$6.016$&$5.158$&$4.541$&$4.040$\\
\hline
& 7&$113.883$&$11.189$&$9.252$&$8.639$&$6.854$&$6.062$&$5.117$&$4.485$&$4.007$\\
\hline
& 8&$121.928$&$11.711$&$9.155$&$8.562$&$6.803$&$5.983$&$5.095$&$4.484$&$3.995$\\
\hline
& 9&$129.452$&$11.947$&$9.028$&$8.525$&$6.790$&$6.003$&$5.069$&$4.470$&$3.986$\\
\hline
\end{tabular}



\end{document}
